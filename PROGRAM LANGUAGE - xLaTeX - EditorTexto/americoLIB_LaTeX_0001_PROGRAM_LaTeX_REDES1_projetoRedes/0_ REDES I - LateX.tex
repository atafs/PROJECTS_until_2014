\documentclass[13pt,a4paper]{report}
\usepackage[utf8]{inputenc}
\usepackage[hmargin=2cm,vmargin=3.5cm,bmargin=2cm]{geometry}
\usepackage{graphicx}

\pagestyle{fancy} %estilo fancy
\lhead{\rightmark} % esquerda do cabeçalho
\chead{} %centro do cabeçalho
\rhead{\leftmark} % direita do cabeçalho
\lfoot{} %esquerda do rodapé 
\cfoot{\thepage} %centro do rodapé
\rfoot{} %direita do rodapé

%dados da capa
\title{REDES DIGITAIS I - parte I do projecto}
\author{Américo Tomás (54149), Ianick Insaly(60585), Eduardo Gonçalves(60596) }
\date{Março 2014}

\begin{document}

\maketitle

\tableofcontents %Índice de conteúdos

\chapter{Introdução}

Sem deteção e correção de erros teríamos sérios problemas de integração e utilização das redes. A quem nunca aconteceu ficar meia dúzia de segundos à espera que uma página na web abra e Nada. Sem abordarmos estes assuntos seriamente, isso aconteceria a todo o instante. 

\paragraph{}
Neste projeto vamos abordar deteção e correção de erros. Sendo importante distinguir claramente a diferença entre ambos. Existem algoritmos que detetam, mas não corrigem. Outros detetam e corrigem, mas poucos erros e assim sucessivamente. Cada um tem vantagens e desvantagens. Sendo a sua utilização muito variante do tipo de utilização que está a ser dado à rede.

Também vamos abordar ritmo binário, bem como valores médios de repetição de envio de PDU's (no nível 2: tramas).

\paragraph{}
Naturalmente o ritmo a que conseguimos aceder a informação, enviando e recebendo a mesma é vital. Saber os atuais limites para tirar o máximo proveito das redes já existentes. Ao mesmo tempo que é possível procurar e planear de forma a criar infra-estruturas, que juntamente com a investigação que está a decorrer a nível mundial, irá permitir dar aos atuais clientes, novas velocidades, maior estabilidades, sempre tendo em consideração o consumo energético.

\paragraph{}
Neste projeto iremos procurar perceber o que é uma rede, bem como lidar com alguns dos desafios inerentes ao seu funcionamento. Desde a criação de expressões gerais para abordarmos estes problemas, testando vários valores. Como posteriormente fazendo a implementação de da mesma.

\paragraph{}
Este parte do projeto foi feita no editor de texto LateX. Muito bom para escrita de linguagem matemática, de forma a provar os assuntos que vamos abordando. Existem anexos (A, B e C), feitos tanto em Excel, como cópias de ecrã para poder analisar os resultados das experiências feitas. 

\paragraph{}
Com este Trabalho pretende-se estudar algumas das funcionalidades que estão presentes nos níveis de ligações de dados no modelo de OSI do ISO, nomeadamente:
O ritmo binário máximo que um cliente pode colocar na interface com o nível ligação de dados, O valor médio para o atraso sofrido por uma mensagem que é colocada na interface do nível ligação de dados e a probabilidade de ser entregue ao cliente destino uma mensagem com erros.
Nesta parte do trabalho iremos apresentar resultados analíticos, recorrendo a um conjunto de expressões e aproximações teóricas.  

%inserir figura
\begin{figure}[!htp]
\centering
\includegraphics[scale=0.2]{redes_introducao.jpg}   
\caption{Legenda}
\end{figure}



%perguntas 2
\chapter{Detecção de erros}

%secção 2.1
\section{Criar o $CRC{_8my}$ do nosso grupo}

Após utilizar o CRC-8my.java, obtivemos para o nosso grupo o seguinte número Binário:
\paragraph{}

\emph{•} Padrão binário gerado: 
$111100011$
\paragraph{}
    
\emph{•} Polinómio: 
$1*x^8+1*x^7+1*x^6+1*x^5+0*x^4+0*x^3+0*x^2+1*x^1+1*x^0$
	
\paragraph{}
\emph{•} Eliminando os elementos a zero:
$(x)=x^8+x^7+x^6+x^5+x^1+1$
    

\paragraph{}    
%secção 2.2  
\section{Expressão da taxa de redundância}  
A taxa de redundância do código para os seguintes algoritmos está apresentado em anexo (A), numa folha de Excel.  
\subsection{cálculo do n:}
	Para obtermos a Taxa de redundância, precisamos primeiro de calcular o r (bit de redundância) para cada um dois algoritmos.
	
\paragraph{}
\emph{•} Bit de paridade: r=1;
	 
\paragraph{}
\emph{•} Hamming:	r = $\log_2 m + 1$; 
	 
\paragraph{}
\emph{•} CRC: r = depende (polinómio de maior grau)\paragraph{}
\rightarrow Sendo:${ \textbf{\textit{n = m + r} }}$ 
	


\subsection{cálculo da taxa:}
\begin{equation}
TaxaDeRedundância =\frac{r}{m + r}
\end{equation}
\paragraph{}
	

	
	
%secção 2.3
\section{Probabilidade de estar ok}
Para obter a probalilidade de uma trama ser enviada sem erros para os 3 tipos de algoritmos, colocamos em anexo (B) a respectiva informação, numa folha de excel.

\subsection{cálculo de probabilidade de estar ok}
	\begin{equation}
	P_{ok} = (1 - P_{eb})^{n}
	\end{equation}
	
\paragraph{}
\emph{•} tendo o $n$ sido calculado anteriormente, variando para cada um dos algoritmos consoante o valores de $r$ dos mesmos.
	\paragraph{}

	 

%secção 2.4
\section{Bit de paridade: expressão exata}

\emph{•} Bit de paridade (d=2), (\textbf{expressão exata}):
    
- numerador: Os somatórios dos pares.	
    	
- denominador: Probabilidade condicionada 
	
\begin{equation}
P_{nd|e}(n,Peb)=\frac{\sum_{(i=par)}^{n} C_{i}^n * P_{eb}^i * (1 - P_{eb})^{n-i}\]}{1-(1-P_{eb})^n}
\end{equation}
	
\paragraph{}
\emph{•} Na expressão exata, o raciocínio usado foi o de calcular todos os valores possíveis, que o algoritmo Par assume que estão certos, mas que podem não estar. Sendo o operador somatório utilizado para representar todos os valores possíveis até $'n'$.
      
\paragraph{} 
%secção 2.5    
\section{Bit de paridade: expressão aproximada}

\emph{•}  Bit de paridade (d=2), (\textbf{expressão aproximada}):

\paragraph{}
$P_{nd|e}^{dh}(n,Peb)\approx \frac{1 - (P_{0} + P_{1})}{1-(1-P_{eb})^n}}$

\paragraph{}
\begin{equation}
P_{nd|e}^{dh}(n,Peb)\approx \frac{1 - [(1 - P_{eb})^n] - [ C_{1}^n * P_{eb}^1 * (1 - P_{eb})^{n-1}]}{1-(1-P_{eb})^n}}
\end{equation}

\paragraph{}
\emph{•} Na expressão aproximada, o raciocínio utilizado foi uma espécie de 'negação'. Ou seja, queremos todos os valores (daí termos o valor $'1'$), subtraindo aqueles que estão certos de certeza (que são os valores da $P_{0}$ e da $P_{1}$). Estamos a usar uma aproximação por Excesso (são praticamente sempre por excesso), pois consideramos todos os casos $'1'$, quando no algoritmo Par apenas nos interessaria os pares. Nesta situação, o operador somatório desapareceu, sendo simplesmente substituído pelo valor $'1'$

\paragraph{}
%secção 2.6
\section{Código de hamming: expressão aproximada}

\emph{•}  Código Hamming (d=3), (\textbf{expressão aproximada}):

\paragraph{}
$P_{nd|e}^{dh}(n,Peb)\approx \frac{1 - (P_{0} + P_{1} + P_{2})}{1-(1-P_{eb})^n}}$

\paragraph{}
\begin{equation}
P_{nd|e}^{dh}(n,Peb)\approx \frac{1 - [(1 - P_{eb})^n] - [ C_{1}^n * P_{eb}^1 * (1 - P_{eb})^{n-1}] - [ C_{2}^n * P_{eb}^2 * (1 - P_{eb})^{n-2}]}{1-(1-P_{eb})^n}}
\end{equation}

\paragraph{}
\emph{•} Na expressão aproximada mas para o Código Hamming, o raciocínio usado é semelhante ao do Bit paridade, com a exceção de o $d = 3$, sendo $d$ a distância de Hamming. Por esse motivo quero todos os valores ($'1'$), exceto o $P_{0}$, $P_{1}$ e $P_{2}$.

\paragraph{}
%secção 2.7
\section{Código de hamming}

\paragraph{}
\emph{•} $G_{4_ITU}(x) = G_{19}(x) = x^4 + x + 1$;
\paragraph{}
\emph{•} $G_{8_CCITT}(x) = G_{263}(x) = x^8 + x^2 + x + 1$;
\paragraph{}
\emph{•} $G_{8_my}(x) = G_{483}(x) = x^8 + x^7 + x^6 + x^5 + x^1 + 1$;

\begin{equation}
$(para valores exatos): $
P_{nd} = \frac{\sum_{(i)}^{n} W_{i} * P_{eb}^i * (1 - P_{eb})^{n-i}\]}{1-(1-P_{eb})^n} 
\end{equation}

\begin{equation}
$(para valores aprox.): $
P_{nd} \approx \frac{\sum_{(i)}^{10} W_{i} * P_{eb}^i * (1 - P_{eb})^{n-i}\]}{1-(1-P_{eb})^n} 
\end{equation}

\paragraph{}
\emph{•} A diferença entre os valores exatos e aproximados está num pedido feito pelo enunciado, para restringirmos a pesquisa a 10 elementos; $W = weight = pesos$;

\paragraph{}
\paragraph{}
\emph{•} Ham_{7,4} \rightarrow $W = 3$;
\paragraph{}
\emph{•} Ham_{31,26} \rightarrow $W = 3$;

\paragraph{}
Logo, a expressão geral aproximada (Hamming: d é sempre igual a 3). Assim, usamos a equação $2.5$ para o cálculo de Hamming.

\paragraph{}
%secção 2.8
\section{cógigo CRC}

\paragraph{}
\emph{•} G_{4-ITU}, $m=4$ \rightarrow $W = erro$; %19
\paragraph{}
\emph{•} G_{8-CCITT}, $m=4$ \rightarrow $W = 4$; %263
\paragraph{}
\emph{•} G_{8-my}, $m=4$ \rightarrow $W = 4$; %483

\paragraph{}
\paragraph{}
\emph{•} G_{4-ITU}, $m=26$ \rightarrow $W = 2$; %19
\paragraph{}
\emph{•} G_{8-CCITT}, $m=26$ \rightarrow $W = 4$; %263
\paragraph{}
\emph{•} G_{8-my}, $m=26$ \rightarrow $W = 4$; %483


\paragraph{}
\paragraph{}
\emph{•} A expressão geral para os $W = 4$ é:

\paragraph{}
$P_{nd}^{dh}(n,Peb)\approx \frac{1 - [(1 - P_{eb})^n] - [ C_{1}^n * P_{eb}^1 * (1 - P_{eb})^{n-1}] - [ C_{2}^n * P_{eb}^2 * (1 - P_{eb})^{n-2}] - [ C_{3}^n * P_{eb}^3 * (1 - P_{eb})^{n-3}] - [ C_{4}^n * P_{eb}^4 * (1 - P_{eb})^{n-4}]}{1-(1-P_{eb})^n}}$ 

%perguntas 3
\chapter{Correção de Erros}

%secção 3.1
\section{Método de correção de erros FEC}

\subsection{Ham(7,4)}

\paragraph{}
\emph{•} $= (1-P_{0}) + (1 - P_{1})$

\begin{equation}
	= (1 - P_{eb})^{7} + + C_{1}^7 * P_{eb} * (1 - P_{eb})^6]
\end{equation}

	
\paragraph{}
\subsection{Ham(31,26)}

\paragraph{}
\begin{equation}
	= (1 - P_{eb})^{31} + C_{1}^{31} * P_{eb} * (1 - P_{eb})^{30}]
\end{equation}
\paragraph{}	


%secção 3.2
\section{}

\subsection{Expressão de tempo de transferência}

\paragraph{}
\emph{•} O tempo de atraso vai ser igual à soma do Tempo de propagação mais o Tempo de transferência da trama.

\paragraph{}
\begin{Large}
\begin{equation}
T_{atraso} = T_{propagação} + T_{transferência}
\end{equation}
\end{Large}	

\paragraph{}
\emph{•} Assumimos que não existe tempo de atraso entre o emissor e o recetor. Se existir um $T_{delay}$, só precisamos de somar mais uma parcela.

\paragraph{}
%secção 3.3
\section{Método de correção de erros ARQ: stopAndWait}

\subsection{expressão média do tempo de atraso (código CRC)}

\paragraph{}
\emph{•} Usando o código CRC, queremos a expressão média do tempo de atraso:

\paragraph{}
$\sum\limits_{i=1}^{+\infty} n * p^n = \frac{p}{(1 - p)^2}$

\paragraph{}
$E[D] = N_{re-transmissões} + T_{transferência}$


\paragraph{}
sendo:
$N_{re-transmissões} = N_{transmissões}$


\paragraph{}
então:
$N_{transmissões} = \frac{1}{P_{sucesso}} + T_{rtt} = 2 * T_{atraso}$

\paragraph{}
\begin{Large}
\begin{equation}
N_{transmissões} = \frac{1}{(1 - P_{ok}) + (1 - P_{nd)})} + (2 * T_{atraso})
\end{equation}
\end{Large}	


%perguntas 4
\chapter{Ritmo binário útil}


\section{Método de correção de erros ARQ: stopAndWait}

%secção 4.1
\subsection{taxa de utilização e ritmo binário útil}

\paragraph{}
\begin{Large}
\begin{equation}
r_{binário} = \frac{L}{T_{tx}}
\end{equation}
\end{Large}	

\paragraph{}
\emph{•} $U = \frac{r_{binário}}{E[D]}$

\paragraph{}
\begin{Large}
\begin{equation}
U = \frac{\frac{L}{T_{tx}}}{N_{re-transmissões} + T_{transferência}}
\end{equation}
\end{Large}	

%perguntas 5
\chapter{Geração de tráfego e filas de espera}

%secção 5.1
\section{Expressão geral}

\paragraph{}
Expressão geral para calcular valor médio na fila em função de $\lambda$ (demonstração):

\paragraph{}
\emph{•} É-nos dado no enunciado que:
$E[w] = \frac{\lambda * (E[D^2])}{2 * (1 - \rho)}$

\paragraph{}
\emph{•} Também sabemos que: 
$N_{T} = \frac{1}{P_{sucesso}} = \frac{1}{(1-P_{eb})}$

\paragraph{}
\emph{•} Por sua vez $(D = delay)$: 
\paragraph{}
$D = T_{rtt} * N_{T} = T_{rtt} * \frac{1}{(1-P:{eb})}$ 	

\paragraph{}
\emph{•} Calculando primeiro $(E[D])$:

\paragraph{}
sendo:
\paragraph{} 
$D_{i} = T_{rtt} * i$; 
\paragraph{}
$P_{i} = P_{eb}^{i-1} * (1-P_{eb})$ 
\paragraph{}
$\sum\limits_{i=1}^{+\infty} i * P_{eb}^i = \frac{P_{eb}}{(1-P_{eb})^2}$

\paragraph{}
\paragraph{}
então:
\paragraph{}
$E[D] = \sum\limits_{i=1}^{+\infty} D_{i} * P_{i} = \sum\limits_{i}^{+\infty} (T_{rtt} * i) * [ P_{eb}^{i-1} * (1 - P_{eb})]$

\paragraph{}
$E[D] = T_{rtt} \sum\limits_{i}^{+\infty} (i) * [ P_{eb}^i * P_{eb}^{-1} * (1 - P_{eb})]$

\paragraph{}
$E[D] = T_{rtt} * \frac{1}{(1 - P_{eb})}$

\paragraph{}
\paragraph{}
\emph{•} Passando para o cálculo de $D^2$: $(E[D^2])$

\paragraph{}
sendo:
\paragraph{} 
$\sum\limits_{i=1}^{+\infty} i^2 * P_{i} = P_{eb} * (\frac{(1 + P_{eb})}{(1-P_{eb})^3)}$

\paragraph{}
\paragraph{}
$E[D^2] = \sum\limits_{i=1}^{+\infty} (T_{rtt} * i)^2 * [P_{eb}^{i-1} * (1 - P_{eb})]$

\paragraph{}
$E[D^2] = T_{rtt}^2 * \sum\limits_{i=1}^{+\infty} (i)^2 * [P_{eb}^{i} * P_{eb}^{-1} * (1 - P_{eb})]$

\paragraph{}
$E[D^2] = T_{rtt}^2 * \sum\limits_{i=1}^{+\infty} (i)^2 * [P_{eb}^{i} * (1 - P_{eb}) * P_{eb}^{-1} * (\frac{P_{eb}}{P_{eb}})]$

\paragraph{}
$E[D^2] = T_{rtt}^2 * P_{eb} * \frac{(1 + P_{eb})}{(1 - P_{eb})^2} * (\frac{1}{P_{eb}})$

\paragraph{}
$E[D^2] = T_{rtt}^2 * \frac{(1 + P_{eb})}{(1 - P_{eb})^2}$


\paragraph{}
\paragraph{}
\emph{•} Para terminar, é dado no enunciado que:
\paragraph{}
$E[w] = \frac{\lambda * (E[D^2])}{2 * (1 - \rho)}$

\paragraph{}
\paragraph{}

\begin{Large}
\begin{equation}
E[w] = \frac{\lambda * (T_{rtt}^2 * \frac{(1 + P_{eb})}{(1 - P_{eb})^2})}{2 * (1 - \rho)}$ 
\end{equation}
\end{Large}
\paragraph{}

%conclusão
\chapter{Conclusão}

\paragraph{}
Concluímos esta primeira parte do projeto, ficando com as experiência calculadas e algumas experiências já feitas. Ao longo do mesmo, ficou claro a importância do bit redundante (r). A distância de Hamming, conceito transversal a vários algoritmos. A utilidade de utilizar probabilidades ao abordar Redes digitais. Pois é uma forma de termos uma ideia de onde vamos investir tempo a resolver problemas, nos concentrando naqueles em que é mais provável acontecerem. 

\paragraph{}
Qual o processo de envio, re-envio, de emissor para receptor. E quais os diferentes tempos envolvidos no processo. Terminando esta parte do projecto com o cálculo de uma expressão para valores médios.

\paragraph{}
Aspetos a concluir são a criação de gráficos em Excel, de forma a visualizar e perceber numa função o comportamento de uma Rede. É uma maneira extremamente visual, que é amplamente usada na área da engenharia para mais facilmente ver onde está um problema. É comum olhar para expressões analíticas. E nem sempre ser capaz de retirar informações delas. Mas os gráficos, bem como regressões lineares, ajudam muito os engenheiros a melhor performance, eficiente e produtividade de sistemas, como é o caso das Redes.

%inserir figura
\paragraph{}
\begin{figure}[!htp]
\centering
\includegraphics[scale=1]{redes_conclusao.jpg}  
\caption{Legenda}
\end{figure}



.
\end{document}
